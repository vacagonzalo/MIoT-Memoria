%%%%%%%%%%%%%%%%%%%%%%%%%%%%%%%%%%%%%%%%%%%%%%%%%%%%%%%%%%%%%%%%%%%%%%%%%%%%%%%%
%
% Template license:
% CC BY-NC-SA 3.0 (http://creativecommons.org/licenses/by-nc-sa/3.0/)
%
%%%%%%%%%%%%%%%%%%%%%%%%%%%%%%%%%%%%%%%%%%%%%%%%%%%%%%%%%%%%%%%%%%%%%%%%%%%%%%%%

%----------------------------------------------------------------------------------------
%	PACKAGES AND OTHER DOCUMENT CONFIGURATIONS
%----------------------------------------------------------------------------------------

\documentclass[
11pt, % The default document font size, options: 10pt, 11pt, 12pt
%oneside, % Two side (alternating margins) for binding by default, uncomment to switch to one side
%chapterinoneline,% Have the chapter title next to the number in one single line
spanish,
singlespacing, % Single line spacing, alternatives: onehalfspacing or doublespacing
%draft, % Uncomment to enable draft mode (no pictures, no links, overfull hboxes indicated)
%nolistspacing, % If the document is onehalfspacing or doublespacing, uncomment this to set spacing in lists to single
%liststotoc, % Uncomment to add the list of figures/tables/etc to the table of contents
%toctotoc, % Uncomment to add the main table of contents to the table of contents
parskip, % Uncomment to add space between paragraphs
codirector, % Uncomment to add a codirector to the title page
headsepline, % Uncomment to get a line under the header
]{MastersDoctoralThesis} % The class file specifying the document structure



%----------------------------------------------------------------------------------------
%	INFORMACIÓN DE LA MEMORIA
%----------------------------------------------------------------------------------------

\thesistitle{Evaluador de microcontroladores para misiones espaciales} % El títulos de la memoria, se usa en la carátula y se puede usar el cualquier lugar del documento con el comando \ttitle

% Nombre del posgrado, se usa en la carátula y se puede usar el cualquier lugar del documento con el comando \degreename
%\posgrado{Carrera de Especialización en Sistemas Embebidos} 
%\posgrado{Carrera de Especialización en Internet de las Cosas} 
%\posgrado{Carrera de Especialización en Intelegencia Artificial}
%\posgrado{Maestría en Sistemas Embebidos} 
\posgrado{Maestría en Internet de las cosas}

\author{Esp. Ing. Gonzalo Nahuel Vaca} % Tu nombre, se usa en la carátula y se puede usar el cualquier lugar del documento con el comando \authorname

\director{Ing. Roberto Cibils (INVAP)} % El nombre del director, se usa en la carátula y se puede usar el cualquier lugar del documento con el comando \dirname
\codirector{Ing. Damian Rosetani (INVAP)} % El nombre del codirector si lo hubiera, se usa en la carátula y se puede usar el cualquier lugar del documento con el comando \codirname.  Para activar este campo se debe descomentar la opción "codirector" en el comando \documentclass, línea 23.

\juradoUNO{Mg. Ing. Iván Andrés León Vásquez (INVAP)} % Nombre y pertenencia del un jurado se usa en la carátula y se puede usar el cualquier lugar del documento con el comando \jur1name
\juradoDOS{??? Rodrigo Cardenas (???)} % Nombre y pertenencia del un jurado se usa en la carátula y se puede usar el cualquier lugar del documento con el comando \jur2name
\juradoTRES{Esp. Ing. Pablo Almada (FIUBA-UTN)} % Nombre y pertenencia del un jurado se usa en la carátula y se puede usar el cualquier lugar del documento con el comando \jur3name

\ciudad{Ciudad Autónoma de Buenos Aires}
%\ciudad{ciudad de Mendoza}

\fechaINICIO{marzo de 2021}
\fechaFINAL{junio de 2022}


\keywords{Sistemas embebidos, FIUBA} % Keywords for your thesis, print it elsewhere with \keywordnames


\begin{document}


\frontmatter % Use roman page numbering style (i, ii, iii, iv...) for the pre-content pages

\pagestyle{plain} % Default to the plain heading style until the thesis style is called for the body content


%----------------------------------------------------------------------------------------
%	RESUMEN - ABSTRACT 
%----------------------------------------------------------------------------------------

\begin{abstract}
\addchaptertocentry{\abstractname} % Add the abstract to the table of contents
%
%The Thesis Abstract is written here (and usually kept to just this page). The page is kept centered vertically so can expand into the blank space above the title too\ldots
\centering
Esta memoria explica el trabajo realizado para INVAP SE en el área de la tecnología aeroespacial.
Se realizó una herramienta que simula los efectos de la radiación cósmica en un microcontrolador.

La herramienta sirve para abaratar el costo de la producción de satélites y aumentar su confiabilidad.
Las simulaciones permiten evaluar técnicas de mitigación de errores y componentes no calificados para uso espacial.
Para realizar este trabajo se valió de la teoría de arquitecturas de microcontroladores y sus protocolos de depuración.
\end{abstract}

%----------------------------------------------------------------------------------------
%	CONTENIDO DE LA MEMORIA  - AGRADECIMIENTOS
%----------------------------------------------------------------------------------------

%\begin{acknowledgements}
%\addchaptertocentry{\acknowledgementname} % Descomentando esta línea se puede agregar los agradecimientos al índice
%\vspace{1.5cm}

%Esta sección es para agradecimientos personales y es totalmente \textbf{OPCIONAL}.  

%\end{acknowledgements}

%----------------------------------------------------------------------------------------
%	LISTA DE CONTENIDOS/FIGURAS/TABLAS
%----------------------------------------------------------------------------------------

\tableofcontents % Prints the main table of contents

\listoffigures % Prints the list of figures

\listoftables % Prints the list of tables


%----------------------------------------------------------------------------------------
%	CONTENIDO DE LA MEMORIA  - DEDICATORIA
%----------------------------------------------------------------------------------------

\dedicatory{\textbf{Dedicado a mi hija Helena}}  % escribir acá si se desea una dedicatoria

%----------------------------------------------------------------------------------------
%	CONTENIDO DE LA MEMORIA  - CAPÍTULOS
%----------------------------------------------------------------------------------------

\mainmatter % Begin numeric (1,2,3...) page numbering

\pagestyle{thesis} % Return the page headers back to the "thesis" style

% Incluir los capítulos como archivos separados desde la carpeta Chapters

\chapter{Introducción general}
\label{ch:introduccion}

Este capítulo presenta de forma breve el contexto del trabajo realizado.
Se explica la problemática solucionada y los conceptos necesarios para la lectura de la memoria.

\newcommand{\keyword}[1]{\textbf{#1}}
\newcommand{\tabhead}[1]{\textbf{#1}}
\newcommand{\code}[1]{\texttt{#1}}
\newcommand{\file}[1]{\texttt{\bfseries#1}}
\newcommand{\option}[1]{\texttt{\itshape#1}}
\newcommand{\grados}{$^{\circ}$}

\section{El espacio como recurso estratégico}
\label{sec:1space}

La explotación del espacio exterior permite mejorar la producción de bienes y servicios.
Además, provee datos de valor científico que incrementan la calidad de vida de las personas \citep{BOOK:spaceage}.
Finalmente, es una fuente de desafíos que impulsa el desarrollo tecnológico.

El marco jurídico que regula la explotación del recurso es el tratado sobre el espacio exterior de 1967.
Del cual, la República Argentina es miembro desde el 27 de enero de ese año.
Los países pueden reclamar cualquiera de las órbitas disponibles pero tienen la obligación de hacer uso dentro de un margen de tiempo determinado.
Luego, si el país no ocupó la órbita pierde el derecho a usarla \citep{BOOK:resurrect}.

El mercado de la explotación espacial está compuesto de estados y empresas.
Siendo las últimas quienes adquirieron el protagonismo en el siglo XXI \citep{ARTICLE:transition}.
El marco jurídico y la competencia en este rubro empuja a las compañías a innovar de forma permanente.
En particular, con nuevas técnicas y tecnologías que permitan abaratar el costo de las misiones.

El cliente de este trabajo es INVAP S.E. y es una empresa de la provincia de Río Negro.
La compañía es modelo en su tipo y realiza proyectos tecnológicos complejos en áreas como: reactores nucleares, satélites y radares.
En la figura \ref{fig:saocom} se puede observar un satélite realizado por la empresa.

\newpage

\begin{figure}[htbp]
	\centering
	\includegraphics[width=\textwidth]{./Figures/invapsaocom.jpg}
	\caption{Satélite SAOCOM\protect\footnotemark.}
	\label{fig:saocom}
\end{figure}

\footnotetext{Imagen tomada de la página oficial de INVAP S.E. \citep{WEBSITE:invap}}

\section{Radiación cósmica y sus efectos}
\label{sec:radiacion}

El sol produce partículas de luz e iones pesados que de forma conjunta se denominan viento solar.
Este fenómeno es atenuado antes de llegar a la superficie del planeta gracias al campo magnético terrestre \citep{WEBSITE:structure_space_radiation}.
Como se puede ver en la figura \ref{fig:viento}, las partículas son desviadas por el campo.
Luego, este queda deformado por el viento solar y se genera una magnetosfera asimétrica.
En la tabla \ref{tab:capasmagneticas} se puede observar las características de la asimétria.

\begin{table}[h]
	\centering
	\caption[Cinturón de Van Allen]{Cinturón de Van Allen \citep{WEBSITE:structure_space_radiation}.}
	\begin{tabular}{l c c}    
		\toprule
		\textbf{Cinturón} & \textbf{Frontera}           & \textbf{Partícula dominante}\\
		\midrule
		Interior          & 1,2 - 2,5 radios terrestres & Protones de alta energía\\		
		Exterior          & 2,8 - 12 radios terrestres  & Electrones de alta energía\\
		\bottomrule
		\hline
	\end{tabular}
	\label{tab:capasmagneticas}
\end{table}

La electrónica de los satélites tiene un alto grado de exposición al viento solar.
Esto significa que la probabilidad de incidencia de una partícula cargada en el circuito es mayor.
La incidencia de una partícula genera una traza densa de pares electrón-hueco en los semiconductores \citep{ARTICLE:velazco}.
Además, es posible que esta ionización cause un pulso transitorio de corriente.

Los efectos de la radiación cósmica sobre el circuito pueden ser transitorios o permanentes.
Los permanentes se deben a la destrucción de una parte del circuito.
Esta destrucción es producto de: el disparo de componentes activos parásitos o la generación de plasma dentro del encapsulado \citep{WEBSITE:effects_on_devices}.
Finalmente, en la tabla \ref{tab:radiacion} se puede ver un resumen de los tipos de errores generados por la radiación cósmica.

En la figura \ref{fig:bitflip} se puede ver el efecto transitorio de la radiación sobre los transistores de un integrado.
En particular, la inyección de un error denominado \emph{bit flip}.
Este se manifiesta como el cambio del valor de un bit en un registro o memoria.
En la esquina inferior izquierda de la figura se pueden observar círculos grises claros y oscuros.
Estos representan los huecos y electrones generados por el efecto Compton y fotoeléctrico.
La línea punteada oblicua representa la traza generada al perturbar los electrones de las uniones covalentes del semiconductor.
Esta traza genera perturbaciones en las junturas y logra disparar las compuertas de los transistores.
Luego, la activación de los transistores cambian los niveles de tensión en las señales de datos y control.
Finalmente, el circuito se normaliza pero con valores invertidos de tensión.

\begin{figure}[htbp]
	\centering
	\includegraphics[width=0.8\textwidth]{./Figures/vientosolar.jpg}
    \caption{Capas magnéticas de la tierra y viento solar\protect\footnotemark.}
	\label{fig:viento}
\end{figure}

\footnotetext{Imagen tomada del artículo \emph{Structure of space radiation} \citep{WEBSITE:structure_space_radiation}}

\begin{figure}[htbp]
	\centering
	\includegraphics[width=0.8\textwidth]{./Figures/bitflip.jpg}
    \caption{Ejemplo simplificado de \emph{bit flip} en un bloque \emph{SDRAM}\protect\footnotemark.}
	\label{fig:bitflip}
\end{figure}

\footnotetext{Imagen tomada del artículo \emph{Effects of space radiation on electronic devices} \citep{WEBSITE:effects_on_devices}}

\vfill

\begin{table}[h]
	\centering
	\caption[Efectos de la radiación cósmica]{Efectos producidos por la radiación cósmica \citep{WEBSITE:structure_space_radiation}.}
	\begin{tabular}{l c c}    
		\toprule
		\textbf{Evento}      & \textbf{Acrónimo} & \textbf{Efecto}\\
		\midrule
		Latch-up             & SEL               & Pico de corriente\\		
		Upset                & SEU               & Alteración de datos\\
		Funtional Interrupt  & SEFI              & Cambios en la configuración\\
		Transient            & SET               & Pico de tensión\\
		Burnout              & SEB               & Activación de transistores parásitos\\
		Gate Rapture         & SEGR              & Generación de plasma de alta densidad\\
		\bottomrule
		\hline
	\end{tabular}
	\label{tab:radiacion}
\end{table}

\newpage

\section{Calificación espacial e iniciativa \emph{new space}}
\label{sec:newspace}

A los efectos vistos en la sección \ref{sec:radiacion} se suman: el estrés mecánico del lanzamiento y los cambios de temperatura en la órbita.
Este ambiente genera la necesidad de utilizar componentes con calificación espacial.
Para que un componente alcance la calificación espacial se debe someter a un largo y costoso proceso de acreditación.
Luego, estos componentes adolecen de un elevado precio y atraso tecnológico frente a los del mercado masivo \citep{ARTICLE:negocio}.

La irrupción del sector privado vista en la sección \ref{sec:1space}, trajo una nueva iniciativa comercial denominada \emph{new space}.
Esta iniciativa busca bajar los costos al utilizar componentes no calificados para su uso espacial.
Además, existe la ventaja adicional de introducir tecnología de vanguardia.


El caso de \emph{Starlink} es un ejemplo de \emph{new space} particular.
Su volumen de satélites lanzados permite realizar conclusiones estadísticas significativas.
En particular, su poca capacidad para cumplir sus objetivos si se mantiene la actual tasa de mortalidad de sus satélites \citep{ARTICLE:cibils}.
En la figura \ref{fig:starlinkdeath} se puede observar que la constelación no logrará alcanzar la población deseada.
 
Al problema de población de \emph{Starlink} se suma la gran cantidad de polución generada.
Los satélites fuera de servicio no pueden ser desorbitados y persisten en forma de \emph{debris}.
Como se puede ver en la tabla \ref{tab:starlinkdebris}, el volumen de basura generado es significativo.

\begin{table}[h]
	\centering
    \caption[Proyección de \emph{debris}]{Proyección de \emph{debris} de \emph{Starlink} \citep{ARTICLE:cibils}.}
	\begin{tabular}{c c c c c}    
		\toprule
        \textbf{Lanzamientos} & \textbf{Satélites} & \textbf{Total lanzados} & \textbf{Población} & \textbf{Debris}\\
		\midrule
        12                    & 60                 & 7200                    & 2704               & 4046\\		
        12                    & 180                & 21600                   & 8105               & 12146\\		
        12                    & 400                & 48000                   & 18007              & 26994\\		
        180                   & 60                 & 108000                  & 40000              & 61200\\		
        60                    & 180                & 108000                  & 40000              & 61200\\		
        27                    & 400                & 108000                  & 40000              & 61200\\		
		\bottomrule
		\hline
	\end{tabular}
	\label{tab:starlinkdebris}
\end{table}

\newpage

\begin{figure}[htbp]
	\centering
    \includegraphics[width=\textwidth]{./Figures/starlinkpopulation.png}
    \caption{Proyección de la constelación \emph{Starlink}\protect\footnotemark.}
    \label{fig:starlinkdeath}
\end{figure}
\footnotetext{Imagen tomada de la publicación de Roberto Cibils \citep{ARTICLE:cibils}.}

Las conclusiones del caso \emph{Starlink} muestran la importancia de tener herramientas para simular el ambiente espacial.
En particular, los efectos de la radiación cósmica para poder probar las técnicas de mitigación de errores seleccionadas.
Finalmente, este trabajo agrega valor al cliente al incrementar la confiabilidad de los satélites y evitar los problemas de la competencia.

\section{Estado del arte}
\label{sec:arte}

La microelectrónica se encuentra en un proceso constante de cambio.
Se incrementa la densidad de integración, la velocidad de los dispositivos y se reducen los niveles de tensión eléctrica.
Este progreso hace que los circuitos sean más vulnerables a la radiación cósmica.

El uso de dispositivos sin calificación espacial hace que los riesgos frente a un error transitorio sean mayores.
Además, la aplicación final de vuelo no suele estar disponible durante la fase de desarrollo de los proyectos.
Esto dificulta aún más la mitigación de errores y aumenta la vulnerabilidad de los sistemas.

La manera tradicional de evaluar las técnicas de mitigación de errores es realizar un ensayo por radiación.
En estos ensayos se utilizan instalaciones en tierra para pruebas de radiación.
Sin embargo, estas instalaciones son costosas y la preparación y ejecución de los ensayos consumen mucho tiempo.

Para lograr un ensayo por radiación, las instalaciones deben producir un rayo de partículas cargadas.
Estas partículas se pueden obtener de:
\begin{itemize}
    \item Aceleradores de partículas: en esta categoría se incluyen ciclotrones y aceleradores lineales.
    \item Decaimiento por fisión: basados en el decaimiento por fisión espontánea de elementos como $Cf^{252}$.
\end{itemize}

En la figura \ref{fig:iones} se puede observar una cámara de iones pesados.
Durante el ensayo se ejecuta una metodología particular que define la actividad en el dispositivo bajo prueba.
Además, se necesita de un sistema que controle y observe el dispositivo bajo prueba durante su exposición a la radiación.
Finalmente, se requiere de personal calificado en el diseño, ejecución e interpretación de estos ensayos.

\begin{figure}[htbp]
	\centering
	\includegraphics[width=\textwidth]{./Figures/heavy_ion_latchup_tests_in_louvain_la_neuve.jpg}
    \caption{Cámara de pruebas de iones pesados\protect\footnotemark.}
	\label{fig:iones}
\end{figure}

\footnotetext{Imagen tomada del sitio web ucl.ac.eu \citep{WEBSITE:heavy_ion}.}

Las estrategias para la inyección de errores pueden ser estáticas o dinámicas.
En las estáticas solo se observa si se producen cambios en valores determinados de memorias y registros.
Las pruebas dinámicas se realizan mientras se activan secuencias de lectura y escritura de memorias.
También incluyen la ejecución de programas para estimular el procesador.

El diseño de un ensayo dinámico necesita determinar la contribución de \emph{SEU} sobre una sección trasversal de memoria.
La cual está relacionada con su tiempo de lectura y escritura (ciclo de trabajo).
La sección trasversal de un programa se puede definir entonces como:

\begin{equation}
	\label{eq:crosssection}
    \sigma_{(SEU)} = \sum{d(R_i) \times \sigma_{R_i}}
\end{equation}

\newpage

Donde:
\begin{itemize}
    \item $ d(R_i) $ es el ciclo de trabajo del elemento de memoria $ R $.
    \item $ \sigma_{R_i} $ es la sección trasversal de memoria obtenida por la metodología estática \citep{ARTICLE:velazco}.
\end{itemize}

Como se puede ver en la ecuación \ref{eq:crosssection} diseñar un ensayo dinámico por radiación es un proceso largo y costoso.
Demanda personal capacitado e instalaciones específicas.
Además, la electrónica de vuelo no suele estar presente durante el desarrollo del proyecto.
Finalmente, este método no permite un control preciso del ensayo.

Otra técnica disponible son los ensayos basados en \emph{software}.
Este método se basa en introducir instrucciones espúreas que generen errores.
Esto genera la problemática de estimar la tasa de error.
Con esta tasa se puede determinar en que posiciones del programa corresponde introducir las instrucciones de error.
La tasa de error se estima como:

\begin{equation}
	\label{eq:errorrate}
    \tau_{SEU} = \sigma_{SEU} \times \tau_{inj}
\end{equation}

Donde $ \tau_{inj} $ es la tasa de incidencia de una partícula cargada.

Esta técnica demanda ciclos de la unidad de proceso y por lo tanto consume más tiempo de ejecución.
Además, cada vez que se altera el código fuente de la aplicación se necesita volver a diseñar una serie nueva de ensayos.

Existe una tercera técnica de ensayo de errores transitorios.
Esta técnica está basada en \emph{hardware}; y en el caso de los microprocesadores, se utiliza una sonda de depuración.
La principal ventaja de este método frente al basado en \emph{software}, es que un mismo ensayo puede ser utilizado para múltiples iteraciones del código fuente.
Esto es posible al suponer un flujo constante de partículas cargadas.
De esta manera, los errores siguen un proceso Poisson homogéneo.
Luego, el intervalo de tiempo entre dos errores transitorios expresa una distribución exponencial.
Este razonamiento está sustentado en la siguiente ecuación:

\begin{equation}
	\label{eq:poisson}
    P(N_{SEU}(t + \Delta t) = N_{SEU}(t) ) = e^{-\sigma \times \phi \times \Delta t}
\end{equation}

El trabajo realizado es una solución del tipo ensayo por \emph{hardware}.
Además, se propuso superar el estado del arte de este método al crear una abstracción para el diseño de ensayos.
Los métodos mencionados presentan compromisos de diseño y están resumidos en la tabla \ref{tab:arte}.

\begin{table}[h]
	\centering
	\caption[Comparación de métodos de simulación]{Comparación de métodos de simulación \citep{ARTICLE:velazco}.}
	\begin{tabular}{l c c c}    
		\toprule
        \textbf{Método}        & \textbf{Eficiencia} & \textbf{Costo} & \textbf{Limitación}\\
		\midrule
        \emph{Software}        & Baja                & Bajo           & Ciclos de CPU\\		
        \emph{Hardware}        & Media               & Medio          & Acceso al integrado\\
        Radiación              & Alta                & Alto           & Control del ensayo\\
		\bottomrule
		\hline
	\end{tabular}
	\label{tab:arte}
\end{table}

\newpage

\section{Alcance del trabajo}
\label{sec:alcance}

El trabajo realizado se divide en dos partes:
\begin{enumerate}
    \item \emph{Firmware} para el dispositivo bajo prueba.
    \item Inyector de \emph{soft-errors} por consola de comandos.
\end{enumerate}

El \emph{firmware} en el dispositivo bajo prueba tiene la misión de validar su funcionamiento.
Esto se logró al verificar cada periférico de interés dentro del integrado.
Luego, se generan reportes periódicos que se envían al inyector por consola de comandos.
En la figura \ref{fig:dutsimple} se puede observar un diagrama en bloques simplificado.

\begin{figure}[htbp]
	\centering
	\includegraphics[width=0.7\textwidth]{./Figures/dutsimple.pdf}
    \caption{Diagrama simplificado del dispositivo bajo prueba.}
	\label{fig:dutsimple}
\end{figure}

El inyector por consola de comandos tiene la función de planificar los ensayos y gestionar la introducción de errores.
En la figura \ref{fig:sisesimple} se puede ver como interactúan las partes del sistema.

\begin{figure}[htbp]
	\centering
	\includegraphics[width=0.7\textwidth]{./Figures/sisesimple.pdf}
    \caption{Diagrama simplificado del sistema de inyección de errores.}
	\label{fig:sisesimple}
\end{figure}


\chapter{Introducción específica}

\label{Chapter2}

En este capítulo se detallan las tecnologías que forman parte del trabajo.
Son productos de terceros que se integran en las herramientas entregadas al cliente.

\section{Arquitectura del dispositivo bajo prueba}
\label{sec:dut}

El trabajo fue realizado para un tipo de microcontrolador específico.
Su diseño forma parte de la familia \emph{Cortex M} de la empresa \emph{ARM}.
En la figura \ref{fig:cortexm} se puede observar un diagrama en bloques de la arquitectura.

Al fabricante del dispositivo bajo prueba se le impone respetar el mapa de memoria y registros del núcleo.
Esto permitió construir un inyector de \emph{soft-errors} genérico.
Finalmente, la herramienta entregada funciona para cualquier integrado de la familia \emph{Cortex M}.

\begin{figure}[htbp]
	\centering
	\includegraphics[width=.7\textwidth]{./Figures/Cortex-M4.png}
    \caption{Diagrama de la arquitectura \emph{Cortex M4}\protect\footnotemark.}
	\label{fig:cortexm}
\end{figure}

\footnotetext{Imagen tomada de la página oficial de \emph{ARM Developers}. \citep{WEBSITE:cortexm}}
\newpage

La arquitectura tiene un módulo que permite programar y depurar el integrado.
Este módulo se denomina \emph{CoreSight} y es propio de los dispositivos \emph{ARM}.
En la figura \ref{fig:coresight} se muestra un diagrama en bloques del módulo.
Sus partes principales son:

\begin{itemize}
    \item \emph{Cross Triggering}: permite conectar y encaminar las señales que utilizan las sondas de depuración.  
    \item \emph{Debug Access Port}: es el puerto físico para conectar la sonda de depuración. Es una implementación de la interfaz de depuración \emph{ARM}.
    \item \emph{Embedded Trace Macrocells}: permite extraer información y controlar el núcleo del dispositivo.
    \item \emph{Instrumentation Trace Units}: permite que una sonda de depuración se conecte con las \emph{Embedded Trace Macrocells}.
    \item \emph{ROM Tables}: sirven para que la sonda de depuración identifique al integrado.
    \item \emph{Self Hosted Debug}: son instrucciones específicas de depuración controladas por un procesador secundario.
    \item \emph{Trace Interconnect}: provee puentes para compartir señales de reloj, alimentación y otras señales comunes.
\end{itemize}

\begin{figure}[htbp]
	\centering
	\includegraphics[width=\textwidth]{./Figures/coresight.png}
    \caption{Diagrama del módulo \emph{CoreSight}\protect\footnotemark.}
	\label{fig:coresight}
\end{figure}
\footnotetext{Imagen tomada del artículo \emph{How to debug: CoreSight basis}. \citep{WEBSITE:coresight}}

\section{Servidores y sondas de depuración}
\label{sec:depuracion}

Una sesión de depuración sirve para observar y modificar el estado de ejecución de un programa.
Esto se logra al leer y modificar los valores en registros del procesador y periféricos.
Además, se necesita de un sistema de disparos por eventos y supervisión de recursos.
Finalmente, la sesión debe detener la ejecución del núcleo de ser necesario.
En la figura \ref{fig:debug} se puede observar un esquema simplificado de una sesión de depuración.

\begin{figure}[htbp]
	\centering
	\includegraphics[width=.8\textwidth]{./Figures/debug.pdf}
    \caption{Conexión de una sesión de depuración.}
	\label{fig:debug}
\end{figure}

\newpage

Las sondas de depuración tienen el objetivo de conectar el \emph{Debug Access Port} con el puerto del ordenador del usuario.
Adaptan los niveles de tensión y los protocolos involucrados.
Luego, permiten realizar una sesión de depuración, programar el dispositivo o verificar el estado de los componentes en la placa.
En la figura \ref{fig:sonda} se puede ver la sonda provista por el cliente.

\begin{figure}[htbp]
	\centering
	\includegraphics[width=.8\textwidth]{./Figures/segger.jpg}
    \caption{Sonda de depuración \emph{Segger J-32}\protect\footnotemark.}
	\label{fig:sonda}
\end{figure}
\footnotetext{Imagen tomada de \url{https://www.digikey.com/}}

Un servidor \emph{On-chip debugger} tiene la misión de abstraer la conexión la sonda de depuración.
Además, facilita el manejo del ciclo de vida de la sesión y permite usar \emph{software} como \emph{GNU Project debugger}.
Finalmente, es la base de una pila de tecnologías que permite el uso de herramientas como \emph{Eclipse IDE}.
En la tabla \ref{tab:servidores} se puede observar un resumen de los servidores evaluados en el trabajo.

\begin{table}[h]
	\centering
	\caption[Servidores de depuración]{Comparativa entre servidores de depuración}
	\begin{tabular}{l c c c}    
		\toprule
        \textbf{Servidor} & \textbf{API} & \textbf{Acceso}   & \textbf{Licencia}\\
		\midrule
        OpenOCD           & tcl                         & Registros y SDRAM & MIT\\        	
        PyOCD             & Python 3                    & Registros y SDRAM & Apache-2.0\\
		\bottomrule
		\hline
	\end{tabular}
	\label{tab:servidores}
\end{table}

\section{Periféricos de interés}
\label{sec:perifericos}

El dispositivo bajo prueba ofrece una variedad de periféricos para el desarrollo de aplicaciones.
Sin embargo, el cliente manifestó interés solo en los que se nombran a continuación:
\begin{itemize}
    \item CAN: este periférico permite al microcontrolador ser el dispositivo principal en una \emph{Controller Area Network}. La red es de grado industrial y fue diseñada para gestionar una red de sensores en un ambiente automotriz.
    \item PIO: es el puerto de entradas y salidas digitales de propósito general. En el caso del dispositivo bajo prueba, el periférico permite usar circuitos anti rebote, \emph{pull-up} y \emph{pull-down} internos. 
    \item SPI: el periférico permite realizar una conexión del tipo \emph{Serial Peripheral Interface}. Esta conexión es sincrónica y solo apta para distancias cortas.
    \item UART: es un periférico que permite conectarse a puertos y controlar dispositivos serie.
    \item Watchdog: el periférico sirve para detectar un error de ejecución y reiniciar el microprocesador.
\end{itemize}

En la tabla \ref{tab:perifericosresumen} se resume la funcionalidad de cada uno de ellos.

\begin{table}[h]
	\centering
	\caption[Resumen de periféricos]{Resumen de periféricos}
	\begin{tabular}{l c c}    
		\toprule
        \textbf{Periférico} & \textbf{Funcionalidad}\\
		\midrule
		CAN                 & Bus de comunicación de grado industrial\\        	
		PIO                 & Entradas y salidas digitales\\
		SPI                 & Interfaz de comunicación sincrónica\\
		UART                & Puerto para dispositivos serie\\
		Watchdog            & Detección de errores y reinicio del integrado\\
		\bottomrule
		\hline
	\end{tabular}
	\label{tab:perifericosresumen}
\end{table}

\newpage

\section{Requerimientos del cliente}
\label{sec:emphuerimientos}

Se realizaron una serie de reuniones con el cliente y se pudo definir los requerimientos del trabajo.
A continuación se enumeran los principales:

\begin{enumerate}
	\item Referentes al inyector por consola de comandos:
		\begin{enumerate}
			\item Generará de una interfaz de usuario.
			\item Permitirá configurar el ensayo a realizar.
			\item Observará la salida del dispositivo bajo prueba.
            \item Inyectará \emph{soft-errors} en el dispositivo bajo prueba.
			\item Persistirá las operaciones, entradas y salidas.
			\item Generará informes del ensayo realizado.
		\end{enumerate}
	\item Referentes al proceso del dispositivo bajo prueba:
		\begin{enumerate}
			\item Verificará el estado de los periféricos del dispositivo bajo prueba.
			\item Detectará si el dispositivo bajo prueba perdió su secuencia.
			\item Generará reportes de estado de periféricos y secuencia.
			\item Permitirá que inyector por consola de comandos configure el alcance de la secuencia.
			\item Permitirá que inyector por consola de comandos maneje el flujo de su secuencia.
		\end{enumerate}
\end{enumerate}

El cliente definió algunas restricciones para el desarrollo del sistema.
Estas se enumeran a continuación:

\begin{itemize}
	\item Utilización de un repositorio con control de versiones \emph{Gitlab}.
	\item Documentación del código con \emph{Doxygen}.
	\item Utilización exclusiva del lenguaje de programación \emph{Python 3}.
\end{itemize}

 
\chapter{Diseño e implementación} % Main chapter title

\label{Chapter3} % Change X to a consecutive number; for referencing this chapter elsewhere, use \ref{ChapterX}

\definecolor{mygreen}{rgb}{0,0.6,0}
\definecolor{mygray}{rgb}{0.5,0.5,0.5}
\definecolor{mymauve}{rgb}{0.58,0,0.82}

%%%%%%%%%%%%%%%%%%%%%%%%%%%%%%%%%%%%%%%%%%%%%%%%%%%%%%%%%%%%%%%%%%%%%%%%%%%%%
% parámetros para configurar el formato del código en los entornos lstlisting
%%%%%%%%%%%%%%%%%%%%%%%%%%%%%%%%%%%%%%%%%%%%%%%%%%%%%%%%%%%%%%%%%%%%%%%%%%%%%
\lstset{ %
  backgroundcolor=\color{white},   % choose the background color; you must add \usepackage{color} or \usepackage{xcolor}
  basicstyle=\footnotesize,        % the size of the fonts that are used for the code
  breakatwhitespace=false,         % sets if automatic breaks should only happen at whitespace
  breaklines=true,                 % sets automatic line breaking
  captionpos=b,                    % sets the caption-position to bottom
  commentstyle=\color{mygreen},    % comment style
  deletekeywords={...},            % if you want to delete keywords from the given language
  %escapeinside={\%*}{*)},          % if you want to add LaTeX within your code
  %extendedchars=true,              % lets you use non-ASCII characters; for 8-bits encodings only, does not work with UTF-8
  %frame=single,	                % adds a frame around the code
  keepspaces=true,                 % keeps spaces in text, useful for keeping indentation of code (possibly needs columns=flexible)
  keywordstyle=\color{blue},       % keyword style
  language=[ANSI]C,                % the language of the code
  %otherkeywords={*,...},           % if you want to add more keywords to the set
  numbers=left,                    % where to put the line-numbers; possible values are (none, left, right)
  numbersep=5pt,                   % how far the line-numbers are from the code
  numberstyle=\tiny\color{mygray}, % the style that is used for the line-numbers
  rulecolor=\color{black},         % if not set, the frame-color may be changed on line-breaks within not-black text (e.g. comments (green here))
  showspaces=false,                % show spaces everywhere adding particular underscores; it overrides 'showstringspaces'
  showstringspaces=false,          % underline spaces within strings only
  showtabs=false,                  % show tabs within strings adding particular underscores
  stepnumber=1,                    % the step between two line-numbers. If it's 1, each line will be numbered
  stringstyle=\color{mymauve},     % string literal style
  tabsize=2,	                   % sets default tabsize to 2 spaces
  title=\lstname,                  % show the filename of files included with \lstinputlisting; also try caption instead of title
  morecomment=[s]{/*}{*/}
}


%----------------------------------------------------------------------------------------
%	SECTION 1
%----------------------------------------------------------------------------------------
\section{Autovalidación del dispositivo bajo prueba}
\label{sec:autovalidacion}

\begin{figure}[htbp]
	\centering
	\includegraphics[width=0.8\textwidth]{./Figures/firmware_detallado.pdf}
    \caption{Diagrama en bloques del firmware de autovalidación.}
	\label{fig:firmwaredetail}
\end{figure}

\begin{figure}[htbp]
	\centering
	\includegraphics[width=\textwidth]{./Figures/firmwareflow.pdf}
    \caption{Flujo del firmware de autovalidación.}
	\label{fig:firmwareflow}
\end{figure}

\begin{figure}[htbp]
	\centering
	\includegraphics[width=0.8\textwidth]{./Figures/canloopback.png}
    \caption{Diagrama de \emph{loopback} del periférico CAN\citep{ARTICLE:dutdatasheet}.}
	\label{fig:canloopback}
\end{figure}

\begin{figure}[htbp]
	\centering
	\includegraphics[width=0.8\textwidth]{./Figures/labo.jpeg}
    \caption{Fotografía del dispositivo bajo prueba.}
	\label{fig:labo}
\end{figure}

\begin{table}[h]
	\centering
	\caption[Estrategias de depuración]{Comparación entre estrategias de depuración}

	\begin{tabular}{l c c}    
		\toprule
        \textbf{Periférico} & \textbf{Validación}       & \textbf{Detección en un ciclo}\\
		\midrule
		CAN                 & Loopback interno          & Si\\		
		PIO                 & Loopback externo          & No\\
		SPI                 & Loopback externo          & Si\\
		UART                & Lógica en firmware        & No\\
		Watchdog            & Lógica en inyector        & No\\
		\bottomrule
		\hline
	\end{tabular}
	\label{tab:perifericos}
\end{table}

\section{Interfaz de programación de aplicaciones}
\label{sec:api}

\begin{figure}[htbp]
	\centering
	\includegraphics[width=\textwidth]{./Figures/debugsession.pdf}
    \caption{Flujo de una sesión de depuración.}
	\label{fig:debugsession}
\end{figure}

\begin{table}[h]
	\centering
	\caption[Funcionalidades abstraidas]{Funcionalidades abstraidas}

	\begin{tabular}{l c c}    
		\toprule
        \textbf{Funcionalidad}     & \textbf{Patrón de diseño} & \textbf{Acceso}\\
		\midrule
		Conexión al integrado      & RAII                      & Público\\		
		Detener el núcleo          & RAII                      & Privado\\
		Registros CORE: read/write & OOP                       & Público\\
		Memoria SDRAM: read/write  & OOP                       & Público\\
		\bottomrule
		\hline
	\end{tabular}
	\label{tab:funcionalidades}
\end{table}

\section{Sistema de inyección de soft-errors}
\label{sec:sise}

\begin{figure}[htbp]
	\centering
	\includegraphics[width=\textwidth]{./Figures/siseblocks.pdf}
    \caption{Diagrama en bloques del sistema de inyección de soft-errors.}
	\label{fig:siseblocks}
\end{figure}

\begin{figure}[htbp]
	\centering
	\includegraphics[width=\textwidth]{./Figures/concurrencia.pdf}
    \caption{Flujo de tareas concurrentes.}
	\label{fig:concurrencia}
\end{figure}

\begin{figure}[htbp]
	\centering
	\includegraphics[width=0.7\textwidth]{./Figures/poisson.png}
    \caption{Gráfico de distribución Poisson\citep{WEBSITE:poisson}.}
	\label{fig:poisson}
\end{figure}

\section{Biblioteca para el desarrollo de ensayos}
\label{fig:biblioteca}

\begin{lstlisting}[language=Python,label=cod:vControl,caption=Ejemplo de aplicación de la biblioteca.]  % Start your code-block

import sise.library as sise

dut = sise.Connection()

# Bit-flip en SDRAM
addr = 0x20400000
bit = 0
res = dut.bitFlipMemory(addr, bit)
print("res:", res)

del(dut)

\end{lstlisting}

% Chapter Template

\chapter{Ensayos y resultados} % Main chapter title

\label{Chapter4} % Change X to a consecutive number; for referencing this chapter elsewhere, use \ref{ChapterX}

En este capítulo se describe la estrategia de pruebas adoptada para determinar que el sistema se comporta de forma esperada.

%----------------------------------------------------------------------------------------
%	SECTION 1
%----------------------------------------------------------------------------------------

\section{Laboratorio remoto}
\label{sec:lab}

Durante las primeras etapas del desarrollo no se disponía en Buenos Aires del dispositivo bajo prueba.
Por esta razón, se montó un laboratorio remoto en San Carlos de Bariloche.
Se dispuso una placa de evaluación \emph{SAM V71 Xplained Ultra} conectada a un ordenador dentro de la red de INVAP S.E.
La conexión entre la placa y el ordenador se logró a través de una sonda de depuración \emph{Segger J-32}.

Para poder acceder al laboratorio remoto que se muestra en la figura \ref{fig:remotelab} se necesitó:

\begin{itemize}
    \item Credenciales de acceso y conexión a la VPN de INVAP S.E.
    \item Crear un túnel SSH con el ordenador remoto.
\end{itemize}

El túnel SSH se generó con \emph{X11 forwarding} habilitado.
De esta manera, se pudo generar ventanas gráficas en el ambiente local.
Además, las operaciones de consola se integraron al ordenador personal con una sesión de Tmux.
Finalmente, se logró implementar una interfaz de control del laboratorio remoto con una apariencia idéntica al ambiente local.

\begin{figure}[htbp]
	\centering
	\includegraphics[width=\textwidth]{./Figures/vpn.pdf}
    \caption{Diagrama en bloques del laboratorio remoto.}
	\label{fig:remotelab}
\end{figure}

Para poder instalar las dependencias y los servidores OCD evaluados, se habilitaron los puertos necesarios que permitieron al ordenador remoto conectarse a los recursos en la Internet.
Sin embargo, algunos recursos debieron ser compilados en el \emph{host} y la transferencia de los \emph{tarballs} se realizó por medio de \emph{Secure Copy Files (SCP)}.

Con el laboratorio remoto montado, se procedió a realizar las siguientes pruebas:
\begin{itemize}
    \item Pruebas de configuración de sondas de depuración y compatibilidad con servidores OCD.
    \item Pruebas de acceso al dispositivo bajo prueba.
\end{itemize}

Las pruebas referidas a la sonda de depuración arrojaron como resultado lo siguiente:

\begin{itemize}
    \item El modo de \emph{boot} de la sonda determina el nivel de acceso al dispositivo bajo prueba.
    \item Para lograr inyecciones de \emph{soft-errors} la sonda de depuración debe poder iniciar en modo \emph{CMSIS-DAP}.
    \item Si la sonda de depuración no se encuentra en modo \emph{CMSIS-DAP}, \emph{PyOCD} solo puede realizar escritura en la memoria \emph{flash}.
    \item Cambiar de modo una sonda de depuración requiere reiniciarla.
        Por lo tanto, no es factible realizar cambios de configuración luego de iniciado un ensayo.
\end{itemize}

Las pruebas referidas al acceso al dispositivo bajo prueba tuvieron los siguientes resultados:

\begin{itemize}
    \item Si no se dispone de un \emph{Device Family Pack (DFP)}, \emph{PyOCD} se conecta al dispositivo bajo prueba y lo identifica como \emph{Generic Cortex-M}.
    \item Bajo la identificación de \emph{Generic Cortex-M} se puede acceder a todos los registros de núcleo.
    \item Bajo la identificación de \emph{Generic Cortex-M} se puede acceder a la memoria \emph{SDRAM} y reconoce como error una posición desalineada.
    \item Bajo la identificación de \emph{Generic Cortex-M} se puede acceder a otras direcciones del integrado solo en modo lectura.
        Si se intenta ingresar en modo escritura no sucede ningún cambio pero el servidor OCD no responde con un error.
        Su respuesta es de operación exitosa, pero no se manifiestan cambios.
\end{itemize}

PyOCD posee un módulo de búsqueda y descarga de DFP, sin embargo, su funcionamiento no es confiable y genera una excepción durante su ejecución.
Se intentó verificar su funcionamiento en otras plataformas y se pudo observar que su desarrollo fue realizado en el lenguaje de programación \emph{Rust}.
Este módulo hizo imposible instalar el servidor OCD en una \emph{single board computer}.
Dado que, su compilador consume una cantidad de memoria que supera el \emph{hardware} disponible en placas como \emph{Raspberry Pi 4B}.
Se pudo verificar que este es el único módulo escrito en \emph{Rust}, pero no es posible desacoplarlo del servidor OCD.
Finalmente, PyOCD tiene una limitación de plataformas compatibles que podría ser sorteada con \emph{cross} compilación.

La única dificultad en el uso del laboratorio remoto se presentó en las pruebas de la sonda de depuración.
Muchas de las pruebas requirieron reiniciar la sonda y esto solo es posible al desconectar el cable \emph{USB}.
La operación debió ser realizada por el co-director de este trabajo.

En la tabla \ref{tab:funcionalidades}, se puede ver un resumen de las funcionalidades del laboratorio remoto.
Las funcionalidades con tres marcas tienen el mismo nivel de servicio que el laboratorio local, las de dos marcas tienen un nivel algo inferior y las que tienen solo una marca tienen un nivel de servicio bajo.

\begin{table}[h]
	\centering
	\caption[Resumen del laboratorio remoto]{Resumen del laboratorio remoto.}

	\begin{tabular}{l c}    
		\toprule
        \textbf{Funcionalidad}             & \textbf{Nivel de servicio} \\
		\midrule
		Carga de binarios en DUT           & ++  \\		
		Comunicación con registro PIP      & +++ \\
		Comunicación con registro Ubuntu   & +++ \\
		Comunicación con debug access port & +++ \\
		Comunicación con UART              & +   \\
		\bottomrule
		\hline
	\end{tabular}
	\label{tab:funcionalidades}
\end{table}

\section{Ensayos de inyector}
\label{sec:testinyector}

El inyector de \emph{soft-errors} se sometió a ensayos en los siguientes ambientes:

\begin{itemize}
    \item Laboratorio remoto.
    \item Laboratorio local.
    \item Dispositivo alternativo \emph{NUCLEO-F429ZI}.
        Este último ambiente se puede ver en la figura \ref{fig:alternativo} y se utilizó para probar si el inyector es genérico.
        En particular, porque utiliza una sonda de depuración distinta.
\end{itemize}

\begin{figure}[htbp]
	\centering
	\includegraphics[width=0.8\textwidth]{./Figures/alternativo.jpg}
    \caption{Dispositivo alternativo \emph{NUCLEO-F429ZI}.}
	\label{fig:alternativo}
\end{figure}

Se generaron múltiples archivos de configuración para definir distintos casos de prueba.
Luego, se corrieron los ensayos en los tres ambientes y se compararon los resultados.
Además, se usó una variante adicional en el dispositivo alternativo.
Se probó el comportamiento del inyector sobre un blanco con \emph{mbedOS}.
Finalmente, en la tabla \ref{tab:resensayos} se puede observar un resumen de los resultados obtenidos.

En la tabla se marca con tres cruces los ensayos que arrojaron resultados sobresalientes, con dos cruces los ensayos que mostraron inconvenientes mínimos y con una cruz los ensayos con resultados insatisfactorios.

\begin{table}[h]
	\centering
	\caption[Resumen de ensayos]{Resumen de ensayos.}

	\begin{tabular}{l c c c}    
		\toprule
        \textbf{Ensayo}                 & \textbf{Lab. local} & \textbf{Lab. remoto} & \textbf{DUT alterno} \\
		\midrule
		Escritura SDRAM                 & +++                 & +++                  & +++ \\
		Escritura registros CORE        & +++                 & +++                  & +++ \\
		Funcionalidades extras          & ++                  & ++                   & +++ \\
		Halt CORE                       & +++                 & +++                  & +++ \\
		Lectura SDRAM                   & +++                 & +++                  & +++ \\
		Lectura registros CORE          & +++                 & +++                  & +++ \\		
		Uso concurrente de puerto serie & +++                 & +                    & +++ \\
        Resume CORE                     & +++                 & +++                  & +++ \\
		\bottomrule
		\hline
	\end{tabular}
	\label{tab:resensayos}
\end{table}

El dispositivo alternativo arrojó los mejores resultados porque PyOCD tiene el \emph{Device Family Pack}.
Por otro lado, el laboratorio remoto tuvo malos resultados en las pruebas de concurrencia.
Esto fue así ya que se debía pedir ayuda al personal de INVAP S.E. cada vez que la sonda necesitaba ser reiniciada. 

\section{Validación con el cliente}
\label{sec:validacion}

La etapa final del proceso de pruebas fue una serie de demostraciones realizadas al cliente.
Luego de cada demostración se indicaban las correcciones a realizar.
Seguidamente, se mejoraba el código y se repetía la demostración.
Estos ciclos de iteraciones tenían una frecuencia de 15 días.
Finalmente, se llegó al cumplimiento total de los requerimientos como se puede ver en la tabla \ref{tab:validacion}

\begin{table}[h]
	\centering
	\caption[Resumen de la validación con el cliente]{Resumen de la validación con el cliente.}

	\begin{tabular}{l c}    
		\toprule
        \textbf{Expectativas}     & \textbf{Cumplimiento} \\
		\midrule
		Acceso a memoria          & +++                   \\
		Acceso al CORE            & +++                   \\
		Biblioteca de ensayos     & +++                   \\		
		Capacidad de bit-flip     & +++                   \\
		Configuración del sistema & +++                   \\
		Distribución de errores   & +++                   \\
		Validación de periféricos & +++                   \\
        Generación de reportes    & +++                   \\
		\bottomrule
		\hline
	\end{tabular}
	\label{tab:validacion}
\end{table}

En la figura \ref{fig:demobitflip} se puede observar una demostración del acceso a memoria SDRAM.
Se puede ver que la terminal está dividida en las siguientes partes:

\begin{itemize}
    \item Sección izquierda: se hizo una demostración paso a paso.
        Primero, se importó la biblioteca dentro del espacio de trabajo.
        Luego, se conectó al dispositivo y se cargó una dirección de memoria y el bit a invertir.
        Seguidamente, se realizó una inversión y se mostró el valor previo y posterior al \emph{bit flip}.
        Finalmente, se cerró la conexión con el dispositivo alternativo.
    \item Sección derecha: se observa el mapa de memoria SDRAM del dispositivo alternativo.
        Se usó para mostrarle al cliente las direcciones de memoria ensayadas.
\end{itemize}

Este ensayo además de demostrar el acceso a memoria SDRAM también ejercita la capacidad de realizar \emph{bit flip}.
Finalmente, el cliente consideró que se habían cumplido todos los requisitos y que el trabajo se encontraba finalizado.

Los ensayos finales se volvieron a reproducir en presencia de un estudiante de la especialización en sistemas embebidos quién actualmente utiliza la herramienta en el marco de su proyecto final.

\begin{figure}[htbp]
	\centering
	\includegraphics[width=\textwidth]{./Figures/demo_bitflip.png}
    \caption{Demostración de acceso a memoria.}
	\label{fig:demobitflip}
\end{figure}
 
\chapter{Conclusiones}
\label{Chapter5}

Este capítulo explica de forma breve el cierre del trabajo realizado, sus logros y trabajo a futuro.

\section{Logros obtenidos}
\label{sec:logros}

El primer logro a destacar es que se pudo mejorar el estado del arte actual.
Su mejora es la capacidad de describir una simulación en términos de la radiación cósmica.
Esto se logró al construir una capa de abstracción de aplicaciones.
Esta, esconde la gestión de la sesión de depuración.
De esta manera, se superaron las técnicas utilizadas en la actualidad.
Luego, el desarrollador solo debe describir los errores a introducir en el dispositivo bajo prueba.
Finalmente, se agrega valor al mejorar el tiempo, costo y claridad de los ensayos.

Un segundo logro a destacar es la recepción que obtuvo este trabajo.
Esta fue muy positiva y con un impacto mayor al esperado.
Ya que en la actualidad, INVAP S.E. se encuentra en proceso de integrar la herramienta en su ambiente de desarrollo de satélites.
Además, el sistema realizado se utiliza dentro del marco de un proyecto final en la Carrera de Especialización en Sistemas Embebidos.

Como tercer logro a remarcar es que el trabajo pudo cumplir las expectativas y requerimientos del cliente.
Además, se cumplieron sin desvíos respecto a la planificación de tiempo y recursos.
Esto fue posible gracias los contenidos de gestión de proyectos que se impartieron durante el desarrollo de esta maestría.
Finalmente, estos son los objetivos más destacados:

\begin{itemize}
    \item Creación de un sistema de inyección de errores transitorios que permita evaluar técnicas de mitigación de errores.
    \item Acceso a los componentes de interés del dispositivo bajo prueba.
    \item Biblioteca para el diseño de ensayos de radiación cósmica en lenguaje \emph{Python 3}.
    \item \emph{Firmware} de autoevaluación del dispositivo bajo prueba que verifique los periféricos de interés.
\end{itemize}

\newpage

\section{Trabajo futuro}
\label{sec:futuro}

La investigación realizada durante la producción del trabajo sugiere que es posible agregar las siguientes funcionalidades:

\begin{itemize}
    \item Conexión entre el código fuente del dispositivo bajo prueba y el inyector de errores transitorios.
        Esto se lograría a través del uso de los símbolos de depuración generados en el binario.
        Con estos símbolos es posible unir un valor del registro \emph{program counter} con una línea en el código fuente.
        Esta funcionalidad permitiría analizar la vulnerabilidad de un segmento de código escrito en lenguaje \emph{C/C++}.
    \item Creación de instrucciones específicas para la inyección de \emph{single event functional interrupt}.
        Estas instrucciones lograrían alterar los registros que definen la configuración de un periférico.
        De esta manera, se podrían realizar pruebas sobre las técnicas de mitigación de este tipo de errores.
        La dificultad a superar es que los registros de los periféricos son propios del dispositivo en particular.
        Para lograr una abstracción genérica se requiere definir el comportamiento de las funciones en tiempo de ejecución.
\end{itemize}

Si se lograran introducir estas funcionalidades al trabajo realizado, el método de simulación de errores transitorio por \emph{software} quedaría obsoleto.
Esto tendría el efecto de modificar el estado del arte; ya que se evitaría el problema que tiene el método al intentar introducir un error cuando el integrado no se encuentra en modo privilegiado.

Como se mencionó en la sección \ref{sec:logros}, este trabajo forma parte de un proyecto final de la Carrera de Especialización en Sistemas Embebidos.
Probablemente se obtengan sugerencias y nuevas ideas durante su desarrollo.
Finalmente, se seguirá con interés la evolución del trabajo en curso.
 

%----------------------------------------------------------------------------------------
%	CONTENIDO DE LA MEMORIA  - APÉNDICES
%----------------------------------------------------------------------------------------

\appendix % indicativo para indicarle a LaTeX los siguientes "capítulos" son apéndices

% Incluir los apéndices de la memoria como archivos separadas desde la carpeta Appendices
% Descomentar las líneas a medida que se escriben los apéndices

%\include{Appendices/AppendixA}
%\include{Appendices/AppendixB}
%\include{Appendices/AppendixC}

%----------------------------------------------------------------------------------------
%	BIBLIOGRAPHY
%----------------------------------------------------------------------------------------

\Urlmuskip=0mu plus 1mu\relax
\raggedright
\printbibliography[heading=bibintoc]

%----------------------------------------------------------------------------------------

\end{document}  
